
\documentclass[
	% -- opções da classe memoir --
	article,			% indica que é um artigo acadêmico
	11pt,				% tamanho da fonte
	oneside,			% para impressão apenas no verso. Oposto a twoside
	a4paper,			% tamanho do papel. 
	% -- opções da classe abntex2 --
	%chapter=TITLE,		% títulos de capítulos convertidos em letras maiúsculas
	%section=TITLE,		% títulos de seções convertidos em letras maiúsculas
	%subsection=TITLE,	% títulos de subseções convertidos em letras maiúsculas
	%subsubsection=TITLE % títulos de subsubseções convertidos em letras maiúsculas
	% -- opções do pacote babel --
	english,			% idioma adicional para hifenização
	brazil,				% o último idioma é o principal do documento
	sumario=tradicional
	]{abntex2}
	
\usepackage[utf8]{inputenc}

\usepackage[brazilian,hyperpageref]{backref}	 % Paginas com as citações na bibl
\usepackage[alf]{abntex2cite}	% Citações padrão ABNT

% ---

% ---
% Configurações do pacote backref
% Usado sem a opção hyperpageref de backref
\renewcommand{\backrefpagesname}{Citado na(s) página(s):~}
\title{A utilização do hidrômetro digital na automação residencial de São Carlos}
\author{
    Leandro A. P. Gasparim \\
    Lucas Ferreira Melo \\
    Tiago Spana \\
    Valmir Jean Zanchim \\
}
\date{2018}

\begin{document}
\maketitle

\begin{abstract}
   
    Com o grande avanço da tecnologia da Internet das coisas (IOT), é possível a construção de um sistema que dê mais transparência, controle  e além de conscientizar o cliente do seu consumo de água.
    
    Hoje é difícil tomar ciência do consumo de água, pois a leitura necessita ser feita de forma analógica e os cálculos tem que ser feito manualmente com auxilio de planilhas ou outras formas de anotações, já houve disponível no mercado um aplicativo que prometia fazer esse controle, porém, era necessário a interversão humana para o seu funcionamento, o que para um grande número de pessoas acaba sendo um empecilho efetuar esse controle.
    
    Esse artigo, tenta mensurar a viabilidade do desenvolvimento, além de mostrar as vantagens e desvantagem de um dispositivo que traria essa transparência do consumo ao cliente de uma maneira totalmente automatiza. 
    
\end{abstract}

\section{Introdução}


\end{document}
