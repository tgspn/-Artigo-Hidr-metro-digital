
\documentclass[
	% -- opções da classe memoir --
	article,			% indica que é um artigo acadêmico
	11pt,				% tamanho da fonte
	oneside,			% para impressão apenas no verso. Oposto a twoside
	a4paper,			% tamanho do papel. 
	% -- opções da classe abntex2 --
	%chapter=TITLE,		% títulos de capítulos convertidos em letras maiúsculas
	%section=TITLE,		% títulos de seções convertidos em letras maiúsculas
	%subsection=TITLE,	% títulos de subseções convertidos em letras maiúsculas
	%subsubsection=TITLE % títulos de subsubseções convertidos em letras maiúsculas
	% -- opções do pacote babel --
	english,			% idioma adicional para hifenização
	brazil,				% o último idioma é o principal do documento
	sumario=tradicional
	]{abntex2}
	
\usepackage[utf8]{inputenc}

\usepackage[brazilian,hyperpageref]{backref}	 % Paginas com as citações na bibl
\usepackage[alf]{abntex2cite}	% Citações padrão ABNT

% ---

% ---
% Configurações do pacote backref
% Usado sem a opção hyperpageref de backref
\renewcommand{\backrefpagesname}{Citado na(s) página(s):~}
\title{A utilização do hidrômetro digital na automação residencial de São Carlos}
\author{
    Leandro A. P. Gasparim \\
    Lucas Ferreira Melo \\
    Tiago Spana \\
    Valmir Jean Zanchim \\
}
\date{}

\begin{document}
\maketitle

\begin{abstract}
   
   Pensando em facilitar o cotidiano das pessoas, bem como dos profissionais que realizam os trabalhos de leitura do consumo de água. Frente à tecnologia e os recursos atualmente disponíveis, analisando a forma como esse tipo de equipamento e forma de trabalho sãos executados, o projeto apresenta um novo modelo de leitura do consumo de água, que permitirá, além do controle pelo próprio morador com leitura em tempo real, a automatização do processo de coleta desses dados por parte das concessionárias. Este novo modelo consistirá em um equipamento de medição com contadores micro-controlados, sistema embarcado e comunicação sem fio. Tudo isso permitirá que o próprio consumidor consulte seu consumo de água em tempo real, analise seu histórico de consumo e ainda identifique possíveis vazamentos mais facilmente. Já a concessionária, acessará estes dados da própria central através da INTERNET e, se não houver comunicação, um profissional poderá se deslocar e fazer a medição próximo do local, até mesmo dentro do próprio veículo.
   Tendo em vista uma possível escassez de água pelo crescimento das cidades e consequentemente o maior desperdício, o consumo de água em residências e indústrias que tem sido alvo de políticas de conscientização e redução.  Este projeto visa mensurar a viabilidade do desenvolvimento e aplicação de um equipamento capaz de monitorar e fornecer dados do consumo de água. A fim de evitar o desperdício dos recursos natural e financeiro, além de apresentar as vantagens e desvantagens de um dispositivo que traria essa transparência do consumo ao cliente de uma maneira totalmente automatizada de operação.
    
\end{abstract}
\newpage
\section{Introdução}

    Nos dias atuais, a leitura do consumo de água por parte da concessionária consiste no deslocamento de um profissional até as residências portando um PDA (assistente pessoal digital) para o lançamento do consumo de água. Muitas residências já possuem o hidrômetro posicionado de forma a facilitar o trabalho do leiturista, porém, ainda é possível encontrar residência com o hidrômetro de local de difícil acesso, e esses hidrômetros com difícil acesso podem ocasionar dificuldade na medição do consumo da água. Como por exemplo há moradias onde há animais ferozes que além de trazer risco para o profissional acaba sendo difícil o processo de medição. Outra preocupação é com o desgaste do profissional, onde, de baixo de chuva ou sol o trabalho deve ser realizado. 
    
    Quando a concessionária necessita realizar o corte do fornecimento de água, os problemas podem acabar sendo maiores, por conta que o morador pode passar a agir de forma violenta ou ameaçadora pondo em risco a vida do profissional. Um outro ponto a ser considerado é que quando a concessionária necessita realizar um racionamento, a mesma é obrigada a realizar a interrupção em todas as residencias, pois não há um meio viável de se realizar a interrupção individual de forma escalável.
    
    Para o consumidor, as dificuldades são ainda maiores, pois mesmo querendo fazer o uso consciente da água, as dificuldades no controle e monitoramento do consumo o desestimula a fazer o acompanhamento dos seus gasto, sendo necessário esperar até a chegada da fatura, para tomar ciência do quanto conseguiu economizar no seu consumo. Além de que, havendo um vazamento de água, o mesmo só será detectado na próxima fatura, sendo impossível interromper o desperdício a tempo.
    
    A grande maioria desses problemas conseguimos solucionar de uma forma automatizada 
    
    Com o grande avanço e popularização da tecnologia \textbf{(evitem frases de efeito e genéricas aqui )}Internet das coisas (IOT), e a maneira como ela se torna mais acessível, possibilita uma transformação nos padrões de vida em sociedade. Olhando para essa disponibilidade, faz-se possível a construção de sistemas que dê mais transparência e controle, além de conscientização ao cliente acerca do seu consumo de água.
    
    Diante desse contexto, os avanços em eletrônica e tecnologias da informação vêm aumentando seu apoio a diversas áreas e possibilitando melhorias na eficiência de diversos setores, onde o uso de dispositivos eletrônicos permite otimizar tarefas que ainda são executadas de forma precária e manual.
    
    Nesta nova era (evitar frases de efeito), onde as possibilidades são inúmeras e as ideias também. É necessário uma analise do que realmente pode ser significativo e de fato causar o impacto desejado e necessário para melhoria à qual foi proposto. (Esse parágrafo aqui está totalmente solto não liga nada com nada).
    
    Pessoal é necessário refazer a introdução e complementar as informações sugeridas. Do jeito que esta vocês não apresentam nem o problema e nem a solução do mesmo.
\newpage
\subsection{Justificativa}

Surgiu a necessidade de elaboração desse projeto de pesquisa para encontrar um melhor controle no consumo de água em residências urbanas alinhado na demora para coleta das informações dos hidrômetros convencionais pela empresa de serviço de água e esgoto na qual pode modelar os dados coletados em relatórios para tomadas de decisão, melhorias na qualidade da água e entrega
de fornecimento aquífero. Evitando desperdícios e encontrando com mais agilidade vazamentos ou pontos de atenção e manutenção.

No Brasil, a hidro medição manual feita pelas empresas de saneamento atualmente dificulta o controle 
pelo consumidor e consequentemente diminui o uso consciente da água.
A ideia proposta visa a aplicação da tecnologia disponível nos dias atuais para essa medição
e ajudar tanto o consumidor como também própria empresa de distribuição de água, fazendo com que o
acesso a tal informação seja disponível para ambos, podendo controlar e verificar remotamente o consumo
de água. 

\section{Objetivos}

\section{Plano de trabalho}
\subsection{Cronograma}

\section{Material e métodos}

\section{Forma de análise dos resultados}




%-- objetivo

% -- justificativa

% referencial

%até 04/10
    
 
\end{document}
